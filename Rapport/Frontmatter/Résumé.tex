\chapter*{Résumé}
\addcontentsline{toc}{chapter}{Résumé}

Le gestionnaire d'emploi du temps est une application web conçue pour générer automatiquement des emplois du temps, tout en satisfaisant les contraintes et les préférences des professeurs. Son objectif principal est de faciliter la communication entre les responsables de filière et les enseignants, en optimisant la planification des cours et en réduisant les conflits d'horaire. Grâce à cette application, la gestion des emplois du temps devient plus efficace et plus transparente, améliorant ainsi la coordination au sein de l'établissement. \\

Dans ce rapport, nous vous détaillerons les différentes étapes
de la réalisation de ce projet ainsi que les outils que nous avons utilisés pour mener à bien notre travail.\\

Nous commencerons par présenter le contexte général du projet, avant de décrire la conception et l’idée principale de l’application. Enfin, nous aborderons l’aspect technique de notre application, en incluant les outils et les technologies que nous avons appris et mis en œuvre.
\newpage
