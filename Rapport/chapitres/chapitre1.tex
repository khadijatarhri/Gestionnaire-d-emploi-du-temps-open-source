\chapter{Contexte général du projet}
\label{chap:Contexte général du projet}
Ce chapitre portera sur la définition du contexte global du projet ainsi que sur ses objectifs.
\newpage
\section{Problématique}
Dans tous les établissements scolaires, la création des emplois du temps est essentielle mais prend beaucoup de temps au chef de filière. Cela s'explique par la nécessité de respecter plusieurs contraintes et préférences. Chaque professeur a des disponibilités spécifiques, ce qui rend la tâche encore plus complexe. Il faut aussi s'assurer que les cours sont bien répartis pour les élèves et coordonner avec les salles disponibles. Tout cela fait de la création des emplois du temps un véritable défi.
\section{Analyse de besoin}
Pour résoudre ce défi , nous allons créer une application web. Celle-ci générera automatiquement des emplois du temps en respectant des contraintes et exigences spécifiques, telles que la disponibilité des enseignants et les préférences de cours. 
\section{Objectifs du projet}
Parmi les tâches nécessaires pour assurer le bon déroulement du projet :
\begin{itemize}
\item Faire une interface de loggin de l'application pour le chef de filiére et les professeurs.
\item Créer une base de données qui contienne les noms des profs, les salles et les modules de l'ENSIAS.
\item Résoudre le probléme des contraintes et des préférences des professeurs.
\item Visualisation des résultats.
\end{itemize}
